\newpage
{\Huge \bf Abstract}
\vspace{24pt}

This project was done at the Cambridge Machine Learning Group, as part of the larger effort to build an \emph{Automated Statistician}.

Given a data set, the Automated Statistician should run different methods to suggest interpretable hypotheses and the potential models to use for this data.

In this project, we have shown that automatic kernel discovery can be achieved for GP classification. We implemented a kernel structure search procedure inspired by previous work on regression by other members of the Group.

The evaluation on synthetic data sets demonstrated that the greedy procedure guided by approximate marginal likelihood can recover the underlying truth in these controlled experiments. The experiments on the four real-world biomedical data sets proved that our procedure achieves predictive performance on par with other recently proposed Bayesian models. Unlike the more complex models such as Additive GPs, our search procedure builds simple and interpretable kernels, similar to those that a human modeler might build when working on the given problem.

We have shown that the constructed kernels reveal interesting and interpretable patterns in the data. To represent these patterns, we developed functionality for visualising the posterior means of the composite kernels' components. These sets of plots correspond to the patterns identified in the data. This is the first step towards producing the automated reports for the end users, which we plan to pursue in further work.


\newpage


\vspace*{\fill}
